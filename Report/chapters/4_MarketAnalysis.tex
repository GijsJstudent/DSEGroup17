\chapter{Market Analysis}
\label{ch:Marketanalysis}

% Establish the competitive cost and volume of the market for the product or for services the product can provide.
% SWOT Analysis

% 3 Pages: NOTICE EDITORS WHEN YOU ARE GOING OVER OR UNDER THIS LIMIT!

Before designing a product, it is important to determine whether there is a market for it and what the product will add to this market. The same holds for a drone for drone shows. In this chapter, the gap on the market for the drone of One Thousand Little Lights will be investigated.
The chapter will start with the use cases and possibilities of the drone show industry in \autoref{sec:marketopportunities}. After that, the competitors that are organising drone shows and the size of the market is estimated in \autoref{sec:currentmarket}. In \autoref{sec:SWOTanalysismarket}, the SWOT analysis for the market is shown and the gap for this project is defined. Finally, in \autoref{sec:targetcost}, a target cost for the drone of One Thousand Little Lights is defined.

\section{Drone Show Possibilities of Use}\label{sec:marketopportunities}

The goal of a drone show is to amaze people, whether this is during a festival, a national holiday or promotion for a company. Many of the drone shows that have been organised were for entertainment purposes, which is in line with the mission need statement to revolutionize this specific industry.

There are several events where drone shows have been applied and they are listed below:
\begin{multicols}{2}
\begin{itemize}[noitemsep,nolistsep]
    \item Brand events
    \item Campaigns
    \item Ceremonies
    \item Concerts
    \item Festivals
    \item National holidays
    \item Sporting events
    \item Theme park shows
\end{itemize}
\end{multicols}

Brand events can be about product launches, company celebrations or other advertisements. Intel has for example organised a drone show in honour of their 50th anniversary \cite{intelanniversary} and Kia for their new logo reveal \cite{guinnesspyro}. Furthermore, drone shows have been used for the Olympic Games, New Year's Eve and so on. Besides large shows for holiday events, drones can also be used indoors during concerts, sport events or music festivals as an addition to the experience. A very novel application of drone shows has been done by Greenpeace on the 11th of June, 2021. Greenpeace made a film using drone swarms in the form of animals as a campaign during the G7 conference in Cornwall. \cite{greenpeace}

\section{Current Market}\label{sec:currentmarket}
The drone show industry has seen enormous innovation and progress in the last six years. Where in 2015 the first world record was set by Intel for having the most UAVs in the air at the same time, which was a 100 drones, Damoda Intelligent Control Technology set a new record with a stunning 3,051 drones in September 2020 \cite{worldrecords}.

In \autoref{tab:competitors}, a list of companies that execute drone shows is presented. There are not many suppliers that can be found online and information about their revenue is limited. This emphasises the young market of drone shows. From the table, it is clear that the number of drones used in a drone show can be far apart for different companies, ranging from 100 to over a 3,000 drones in a constellation. Drone shows using more than a 1,000 drones are less frequent and more often used to break world records. Deducted from the websites of the suppliers, it seemed that between 100 - 500 drones is the most common number of drones. Especially 300 drones are often used as there are enough drones available to make a stunning animation, but the cost is lower than 500 drones \cite{intel}. It is also not known of every supplier what the duration of their show is, but the longest animation performed by UAVs had a duration of 26 minutes and 19 seconds \cite{guinnesslongestshow}. Most drone shows are shorter than this, e.g. shows from Intel have a duration of 11 minutes \cite{intel} and the current shows from Anymotion Productions take around 15 minutes. 


Most of these companies only provide light shows, but Skymagic and CollMot also provide the opportunity to launch so called pyrodrones. These drones have the opportunity to launch a firework fountain. Pyrotechnics are a new addition to drone shows. The car company Kia set a world record by using 303 pyrodrones for the reveal of their new logo in October 2020 \cite{guinnesspyro}. An impression can be seen in \autoref{fig:Kia_pyro}. Firing pyrotechnics from the drone is also a requirement for the drone of One Thousand Little Lights, set by the customer.

All companies have the possibility to perform shows outside, but some also provide shows indoors, during concerts or theatre plays. These are Damoda Intelligent Control Technology, Dronisos and Skymagic. It is also a requirement for this project to be able to perform indoor drone shows with a minimum of twenty drones.


\begin{table}[h]
\centering
\caption{Drone show competitors}
\label{tab:competitors}
\resizebox{\textwidth}{!}{%
\begin{tabular}{|l|l|r|l|l|}
\hline
\textbf{Drone show companies}                                                    & \textbf{Country}                                                         & \multicolumn{1}{l|}{\textbf{\begin{tabular}[c]{@{}l@{}}Max. number \\ of drones\end{tabular}}} & \textbf{Drone possibilities}                                                                   & \textbf{Website}                                                       \\ \hline
Anymotion Productions                                                            & Netherlands                                                              & 100                                                                                            & Outdoor drones                                                                                 & \cite{anymotion}                    \\ \hline
AO Technology                                                                    & \begin{tabular}[c]{@{}l@{}}Germany, \\ United Arab Emirates\end{tabular} & 1000                                                                                           & Outdoor drones, payload capability                                                             & \cite{AOtech}            \\ \hline
CollMot Entertainment                                                                     & Hungary                                                                  & 100                                                                                            & \begin{tabular}[c]{@{}l@{}}Outdoor drones, aerial image projection, \\ pyrodrones\end{tabular} & \cite{collmot}                      \\ \hline
\begin{tabular}[c]{@{}l@{}}Damoda Intelligent \\ Control Technology\end{tabular} & China                                                                    & 3000                                                                                           & Outdoor drones, indoor drones                                                                  & \cite{damoda}                   \\ \hline
DroneShow Events                                                                 & Netherlands                                                              & 100                                                                                            & Outdoor drones                                                                                 & \cite{droneshowevents} \\ \hline
Dronisos                                                                         & France, United States                                                    & 200                                                                                            & Outdoor drones, indoor drones                                                                  & \cite{dronisos}                 \\ \hline
Geoscan                                                                          & Finland, Russia                                                          & 2000                                                                                           & Outdoor drones                                                                                 & \cite{geoscan}                     \\ \hline
Intel                                                                            & United States                                                            & 1000                                                                                           & Outdoor drones                                                                                 & \cite{intel}         \\ \hline
Skymagic                                                                         & \begin{tabular}[c]{@{}l@{}}Singapore, \\ United Kingdom\end{tabular}     & 300                                                                                            & \begin{tabular}[c]{@{}l@{}}Outdoor drones, indoor drones, \\ pyrodrones\end{tabular}           & \cite{skymagic}                    \\ \hline
\end{tabular}%
}
\end{table}

\begin{figure}[h]
    \centering
    \includegraphics[width=0.7\textwidth]{Figures/Kia_pyro.png}
    \caption{Kia set a world record with 303 pyrodrones. \cite{guinnesspyro}}
    \label{fig:Kia_pyro}
\end{figure}



\begin{comment}
The objective of the Market Analysis is to establish the competitive cost and volume of the market for the
product or for services the product can provide. This is derived from the current market for comparable
products or services and an assessment of the added value of new technology or solutions that could be
implemented. Aspects that should be included are the prediction of future markets, the establishing of
new markets, and the share you foresee to obtain in these markets. Proper market segmentation may
help to address these aspects. In addition, a SWOT analysis (Strengths, Weaknesses, Opportunities,
Threats) should be included in the market analysis as well.
A first version of the Market Analysis is produced for the Baseline Report. Its major purpose is to establish
a target cost for the system or service, and to generate desired functions and requirements based on the
cost and other demands from the market to be competitive. For the Final Report the Market Analysis is
updated for the actual characteristics and estimated cost of the product or service.
\end{comment}


% How will these markets develop in the coming years?
\subsection{Replacement of Fireworks Shows}
In only the last three days of the year fireworks worth over 77 million euro are sold in the Netherlands. However, with the prohibition on fireworks in 2020 because of COVID-19 alternatives needed to be sought. The prohibition was installed to contribute relieving the burden on employees in health care and the police, as fireworks cause accidents every year, are polluting and make a lot of noise. During the New Year's Eve of 2019-2020 1,300 accidents have been reported in the Netherlands, which costs society around €3.2 million every year \cite{vuurwerkongevallen}.  An innovative, exciting and much more sustainable alternative is the use of drone shows, which have already been used for New Year's Eve in Rotterdam in 2020 \cite{droneshowrotterdam}. 
You cannot compare organised drone shows with the public setting of fireworks in terms of safety, but the pollution and noise also hold for organised fireworks shows.

Drones are very suitable for replacing fireworks and are less polluting and noisy. With the increase in fireworks regulations and decrease in drone costs the market for drone shows used for entertainment purposes has a large probability to increase, especially when looking at the high number of revenue that is already made each year in this sector.  



\subsection{Advertisement}\label{subsec:advertisement}

The advertisement market is worth an estimated 545 billion euro in 2021 \cite{Ad_market_total}, out of which 55 billion come from physical outdoor advertising \cite{Ad_market_outside}. Consumer brands finance incredibly  expensive advertising events to get as much consumer attention as possible. Drone shows provide excellent opportunity to display companies' advertisements in a way that is unfeasible to achieve with other methods. It should be noted that a drone show has a very short duration compared to a billboard, which makes it difficult to exactly compare the efficiency of both methods of advertising. Currently drone shows do not occur very often however, so the company can get additional media attention for free which might increase the reach of the ad.      

\section{Market Gap for One Thousand Little Lights}\label{sec:SWOTanalysismarket}
This section first shows the SWOT analysis, including an explanation. After that, the opportunities specific for the drone of One Thousand Little Lights is discussed. The SWOT analysis of the market is shown in \autoref{tab:SWOTmarket}. It can be seen from the table that the strengths and opportunities mainly focus on the possible uses of drone shows, while the weaknesses and threats describe mostly reasons for the product to be not profitable or actually unable to be used (on a frequent basis). The green texts are specific for the drone of One Thousand Little Lights and will be explained afterwards.


% explain the SWOT specific for our drone

\begin{table}[h]
\centering
\caption{SWOT market analysis (green is specific for Starling)}
\label{tab:SWOTmarket}
\begin{small}
\begin{tabular}{|c|l|l|}
\hline
\multicolumn{1}{|l|}{}     & \multicolumn{1}{l|}{\textbf{Helpful}}                      & \multicolumn{1}{l|}{\textbf{Harmful}}                \\ \hline
                           & \cellcolor[HTML]{acec97}\textbf{Strengths}        & \cellcolor[HTML]{ffb6b3}\textbf{Weaknesses} \\ \cline{2-3}
                           & - Custom made drones                                & - Operational difficulties                    \\
                           & - More sustainable than firework shows              & - High initial costs                          \\
                           & - Low noise emission                                        & - Many safety measures                        \\
                           & - Unique advertisement possibilities                                        &  - Damage during transport                       \\
                           & {\color[HTML]{036400} - Modular payload capability} &      - Logistical challenges                \\ 
                           & {\color[HTML]{036400} - Ease of maintenance} &         {\color[HTML]{036400} - No reputation}            \\ 
                & {\color[HTML]{036400} - Stackable drones for mass transport} & \\
\multirow{-9}{*}{\textbf{Internal}} & {\color[HTML]{036400} - Autonomous charging}        &       \\ \hline
                           & \cellcolor[HTML]{acec97}\textbf{Opportunities}    & \cellcolor[HTML]{ffb6b3}\textbf{Threats}    \\ \cline{2-3}
                           & - Multiple use cases                                & - Future government relations                 \\
                           & - Young market                                      & - Future competition                          \\
                           & - Replacement of or addition to fireworks shows                    & - Too expensive for customer                  \\
                           & - Low number of competitors                         &     - Dependent on low number of shows                                        \\
                           & - High demand                                       &                                             \\
\multirow{-7}{*}{\textbf{External}} & - Attention of the media                            &                                             \\ \hline
\end{tabular}
\end{small}
\end{table}

A short explanation might be necessary for some entries in the table:

\begin{itemize}[noitemsep, nolistsep]
\item \textbf{Unique advertisement possibilities:} drone shows provide new possibilities for advertisement, as written in \autoref{subsec:advertisement}.
\item \textbf{Operational difficulties:} performing choreography with many drones in the air requires detailed planning and organisation, good communication and software in the drone.
\item \textbf{High initial costs:} To start a company organising drone shows, there are a lot of initial costs involved as you need to buy many drones before you are able to organise a show.
\item \textbf{Many safety measures:} During a drone show many safety measures are needed, primarily to avoid human errors. These cause the majority of accidents. All safety measures are time-consuming and cost money to comply with. 
\item \textbf{Logistical challenges:} The organisation of a drone show also brings logistical challenges. One can think about transporting a large number of drones or putting them in a grid and calibrate them while on a tight schedule.
\item \textbf{Attention of the media:} As drone shows are not very common, there is often a news article written about the occasion which is beneficial for brand awareness.
\item \textbf{Dependent on low number of shows:} A drone show company can host only a very limited number of shows per year. According to their websites, most of them need approximately 2 or 3 months to prepare a show \cite{intel}, \cite{anymotion},  \cite{geoscan}. This means that only 4-6 shows per year can be performed and that all profit should come from these shows. If a company does not get an order, it can endanger their annual profit. This could be the reason that Intel \cite{intel} and Geoscan \cite{geoscan} also have other forms of income.
\end{itemize}

Next to these, there are also some strengths and weakness coloured green. This is because they are specific for the design for the drone of One Thousand Little Lights. Most of them follow from requirements. 
\begin{itemize}[noitemsep, nolistsep]
     \item \textbf{Modular payload capability:} from a meeting with the client, Anymotion Productions, it became clear that the lifetime of a drone is often dictated by its payload. After 2-3 years the quality of the payload, e.g. LED lights, is not state-of-the-art anymore and should be replaced. Without a modular payload capability this means that the entire drone should be replaced instead of just the LED light. If this possibility is achieved, it is an enormous improvement to currently existing drone designs.
     AO Technology already has drones with the option to carry a payload of 500 grams, but the possibilities have not been specified in further detail \cite{AOtech}. 
     \item \textbf{Ease of maintenance:} it is required by the customer that a one-day training shall be sufficient to replace parts of the drone. This will be taken into account in the subsystem design. The actual maintenance procedure can only be determined in the final design phase. 
     \item \textbf{Stackable drones for mass transport:} Another requirement on the drones is to be stackable. This is useful to make transport processes more efficient. Intel and Geoscan also use this feature together with carrying structures, to make carrying by hand easier and quicker \cite{intel},\cite{geoscan}. 
     \item \textbf{Autonomous charging:} The last requirement that was given was that the drones should have the possibility of charging autonomously via their landing pad. This can increase the duration of drone shows or make it possible to have multiple drone shows quickly after each other.
    \item \textbf{No reputation:} As the drone that is to be designed will be new on the market, there will be a possibility that companies are hesitant with buying this drone. This could be because the product is very new and does not have a reputation of safety or good choreography yet. This is a weakness that is hard to overcome in the beginning, but should diminish after successful performances have been held.   
    \end{itemize}

When looking back at \autoref{tab:competitors}, it can be seen that companies already have many possibilities for drone show performances. They have the capabilities to host large shows, indoor shows, pyrotechnical shows, stack the drones etc. However, all these features have not been combined in one drone yet. Usually different drones are needed for different purposes. This is what the design of One Thousand Little Lights will try to achieve: with the modular payload capability only one drone is needed for all different applications. This makes the use of the drone versatile. Also the possibility of an autonomously charging drone via the landing pad is something that does not exist yet. 

The aim of One Thousand Little Lights is therefore to focus the drone design on ease of operation and versatility. It should have the feature to execute choreography with at least 300 drones as this number is frequently used and perform a drone show of at least 15 minutes to be able to have a similar or longer show time than other drone shows.

\section{Target Cost}\label{sec:targetcost}
This section determines a target cost for the drone and looks at specifications of currently used drones. 

Anymotion Productions provided the information that the purchase costs of the drone they use is €1,500. This does not include maintenance, which they estimated to be between €100 and €200 per motor per drone\footnote{Personal communication with N. Cornelissen (Creative Manager at Anymotion Productions),  11/06/2021. \label{APemail}}. The requirements that were set up earlier and can be revised in 
\autoref{tab:costreqmarket} ensue a total cost of €1,650 per drone, which is very similar to the €1,600 - €1,700 Anymotion Productions currently pays for their drones. If these requirements are met, this will also meet part of the project objective statement regarding an 'economically competitive drone'. 

\begin{table}[h]
\centering
\caption{Requirements related to the financial overview.}
\label{tab:costreqmarket}
\begin{small}
\begin{tabular}{|l|l|}
\hline
\textbf{TAG} & \textbf{Requirement} \\ \hline
COST-AP-1 & The drones shall cost no more than €1000,- per piece. \\ \hline
COST-AP-2 & The expected cost of replacing parts in 1000 light shows shall be no more   than €650,-. \\ \hline
\end{tabular}
\end{small}
\end{table}


\autoref{tab:marketspecs} shows some specifications of two light show drones that were found on the internet. The drone from Sparkl is the same drone that is used by Anymotion Productions. For the design of the drone, it is beneficial if the specifications are similar or better than the drones provided in the table for a similar price to compete in the market.


% 'Market specs'
% This is what our drone should be similar with?

\begin{table}[h]
\centering
\caption{Drone specifications of Sparkl \cite{Sparkl} and UVify \cite{UVify}.}
\label{tab:marketspecs}
\resizebox{\textwidth}{!}{%
\begin{tabular}{|l|l|l|l|l|l|}
\hline
\textbf{Specifications}                       & \textbf{Sparkl} & \textbf{UVify IFO} & \textbf{Specifications}       & \textbf{Sparkl} & \textbf{UVify IFO} \\ \hline
Dimensions without propellers (cm)            & 40 x 40         & 27.5 x 27.5        & Max. control range (m)        & 500             & 1000               \\ \hline
Dimension with propellers (cm)                & 45 x 45         & 40 x 40            & Max. sustained wind speed (kts)       & 25              & 15                 \\ \hline
Height (cm)                                   & 15.5            & 12.5               & Vertical hover accuracy (m)   & 0.1             & 0.1                \\ \hline
Weight (g)                                    & 1103            & 635                & Horizontal hover accuracy (m) & 0.1             & 0.1                \\ \hline
Weight (incl. battery) (g)                    & 1706            & 1050               & RGB led (W)                   & 10              & 27                 \\ \hline
Max. flight time hovering (min)                        & 25              & 25                 & Light strength (lumen)        & 550             & 840                \\ \hline
Max. airspeed (km/h)                          & 72              & 60                 & Battery type                  & Lipo 4S         & Lipo 4S            \\ \hline
Max. operational altitude above sea level (m) & 1500            & 500                & Battery capacity (mAh)        & 6750            & 4200               \\ \hline
Max. control range (m)                        & 500             & 1000               & Battery voltage (V)           & 14.8            & 14.8               \\ \hline
\end{tabular}%
}
\end{table}

To conclude the market analysis, the design should focus on versatility, cost and ease of operations, for companies to be willing to buy this drone.






