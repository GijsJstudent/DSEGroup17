\chapter{Introduction}
\label{ch:introduction}

% 1 page: NOTICE EDITORS WHEN YOU ARE GOING OVER OR UNDER THIS LIMIT!

The task to design a drone that is specifically designed to be used in drone shows was given by Anymotion Productions, a company that performs these shows \cite{projectguide}. This assignment, called project 'One Thousand Little Lights', will be taken on by ten aerospace students of the Delft University of Technology. As concluded in the first report of this project, the mission need statement is as follows:
\begin{itemize}[noitemsep,nolistsep]
    \item \textit{To revolutionize the airborne, audio-visual entertainment industry by 2025.\cite{projectplan}}
\end{itemize}

The project objective statement was decided upon as well:

\begin{itemize}[noitemsep,nolistsep]
    \item \textit{Design an economically competitive, safe and sustainable drone for indoors and outdoors light shows, in 10 weeks, for Anymotion Productions} \cite{projectplan}.
\end{itemize}

After the second report (the baseline report \cite{baseline-report}) and the third (the mid-term report \cite{midterm}) this report presents the detailed design phase. In the baseline report several analyses were performed on the following aspects: functions, market, sustainability and risk. From these analyses all system requirements were derived. Then a design option tree was created and finally five different design concepts were introduced. In the mid-term report different aspects of the drone and the drone show were considered separately to finally choose one of the five design concepts. Besides, smaller trade-offs were made by each department to narrow down the design options, were made visible in the design option tree. The remaining design options are further investigated and developed in this report. The goal of the detailed design phase is to create the most optimal drone design regarding the customer's needs and finalize the product.

The structure of this report is as follows: The progress made so far during the project are presented in \autoref{ch:ProjectProgress}. The functions of the system are revised in \autoref{ch:functionalanalysis}. A market analysis is conducted in \autoref{ch:Marketanalysis}. The design approach for all subsystems is similar and is discussed in \autoref{ch:BudgetBreakdown}, along with the budget breakdown. The subsequent chapters are dedicated to the separate subsystem designs, starting with aerodynamics and propulsion in \autoref{ch:propulsion}, power in \autoref{ch:power}, communication, control and electronics in \autoref{ch:cce}, structures in \autoref{ch:structures} and finally the operations subsystem in \autoref{ch:operations}. After designing the subsystems separately they were integrated into one product, which is explained in \autoref{ch:finaldesign}. The final design is analysed in \autoref{ch:systemanalysis}. This contains analyses on performance, RAMS, technical risk and sustainability. Then a production plan is presented in \autoref{ch:productionplan}, logistics and safety are covered in \autoref{ch:logistics}, a financial overview is given in \autoref{ch:finance} and a system-wide verification and validation is conducted in \autoref{ch:systemverificationandvalidation}. Finally, the report ends with a discussion about potential post-DSE activities in \autoref{ch:postdseactivities} and a conclusion in \autoref{ch:conclusion}. Next to that, a technical specification sheet of the final design can be found in \autoref{AppendixB}.