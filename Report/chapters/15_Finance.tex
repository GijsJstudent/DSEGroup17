\chapter{Financial Overview}
\label{ch:finance}

% 3 pages: NOTICE EDITORS WHEN YOU ARE GOING OVER OR UNDER THIS LIMIT!   

% 31. Return on investment
% In order to establish the Return of Investment (RoI) the following data related to the product shall be known
% or established:
% − Market price of the product (class),
% − Market volume for the product (class),
% − Achievable market share for the product,
% − Development cost of the product,
% − Production cost of the product.
% − Direct operational cost
% A positive balance between number of products sold and the total cost divided by the total cost is the RoI.
% The RoI is established at the end of the project after the cost estimate, and is presented in the Final Report.

% 21. Cost breakdown structure
% The Cost Break-down Structure (CBS) contains the cost elements of the post-DSE project activities. It has
% the shape of an AND tree and serves to identify all elements that contribute to the overall development and
% production cost of the product or system, for which a preliminary design has been produced during the
% DSE. It reflects the cost related to the activities defined in the PD&D logic. It is the basis of the cost estimate
% for the product of system. Typical examples of “standard” CBS’s may be found in literature. The CBS is
% included in the Final Report.
% This should probably be in the Post DSE chapter.

\section{Requirements}

\begin{table}[H]
\caption{Requirements related to the financial overview.}
\label{tab:costreq}
\begin{tabular}{|l|l|}
\hline
\textbf{TAG} & \textbf{Requirement} \\ \hline
COST-AP-1 & The drones shall cost no more than €1000,- per piece. \\ \hline
COST-AP-2 & The expected cost of replacing parts in 1000 light shows shall be no more   than €650,-. \\ \hline
\end{tabular}
\end{table}