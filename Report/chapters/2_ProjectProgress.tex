\chapter{Project Progress}
\label{ch:ProjectProgress}

% Content:
% Project objectives
% Overview of chosen concept
% Overview of subsystem decisions

% 3 pages: NOTICE EDITORS WHEN YOU ARE GOING OVER OR UNDER THIS LIMIT!


In this chapter an overview of the project until now will be given. This will be done by first stating the objectives with which we started the project. Then an overview of the chosen concept and the decisions that have already been made will be given. All of this can be found in more detail in the baseline and midterm reports \cite{baseline-report} \cite{midterm}. 

\section{Project objective}
\label{sec:projectobjective}

In the project 'One Thousand Little Lights' a drone will be designed, which is optimised for using in air shows. The entire mission need statement was concluded to be: \textit{To revolutionize the airborne, audio-visual entertainment industry by 2025.} The aim of this project was summarised in the project objective statement: \textit{Design an economically competitive, safe and sustainable drone for indoors and outdoors light shows in 10 weeks, for Anymotion Productions.} Requirements were set up to make sure these goals are reached. These requirements can be found in the beginning of each chapter about each subsystem of the drone. The driver requirements were mainly focused on the flight time of the drone. The killer requirements were focused on the aspect which will make the drone revolutionary. This consists of the modular payload the drone has, including the fact that it is able to carry pyrotechnics. Another important aspect is that the drone is able to perform indoor and outdoor shows, and that the manufacturing and maintenance cost are competitive on the market. 

\section{Concept overview}
\label{sec:conceptoverview}

In this report the entire detailed drone will be designed in subsequent chapters. Before this in the midterm report, a design concept was already chosen and some decisions in the subsystems were already made \cite{midterm}. An overview of these decisions will be shown in this section.

The concept has a single configuration for indoor and outdoor shows, has a brushless motor and consists of four unconnected and fixed arms. The drone will be made out of polymer thermoplastics. The design is specially shaped for packaging and the landing gear and propellers are removable. The drone has two propellers per arm and two blades per propeller. A Lithium-ion polymer (LiPo) battery is used and the positioning system is split into an indoor and outdoor option. For the indoor shows Ultra-Wideband is used and for the outdoor shows GPS with Real Time Kinematic is utilized. The communication with the ground station is performed via Wi-Fi in case of emergencies, but the show choreography is already fully programmed on the drone beforehand. An extra feature of the drone, next to normal wired battery charging, is the ability to be able to be conductively charged by the landing pad.